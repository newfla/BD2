\chapter{Conclusioni}
I risultati ottenuti sono chiaramente legati all'architettura hardware e software utilizzata per l'esecuzione delle procedure. Durante l'intero progetto è stata utilizzata una piattaforma con le seguenti specifiche:
\begin{itemize}
	\item Xiaomi Notebook Air 13
	\item Intel Core i7
	\item SSD da 256GB
	\item RAM da 8GB
	\item Windows 10
\end{itemize}
I tempi sono necessariamente soggetti a limiti fisici e rumori dovuti a:
\begin{enumerate}
	\item Capacità hardware del calcolatore
	\item Scheduling del sistema operativo.
	\item Parametri di configurazione del DBMS.
\end{enumerate}
Per ammortizzare la perturbazione, le fasi di cronometraggio prevedono l'acquisizione di più campioni dello stesso tipo, dai quali si è poi ricavata la media. I valori ottenuti vanno inoltre scalati rispetto alla dimensione del dataset. A tal proposito si è simulata una cardinalità massiccia di acquisizioni (circa 1.5 milioni).\\
A valle dell’analisi condotta si è arrivati alle seguenti riflessioni:
%- Aumentare la quantità di memoria principale a disposizione per le operazioni di raggruppamento, ordinamento e join può sensibilmente migliorare l’esecuzione di alcune query, permettendo l’utilizzo di operazioni completamente in-memory.

\begin{itemize}
	\item Avere i dati precalcolati può avere un impatto notevole a favore dell’esecuzione della query, nonostante il numero di accessi alle pagine di memoria sia maggiore, almeno quando il calcolo è relativamente complesso e dipendente da più misure.
	Di contro, avere dati precalcolati, porta ad un aumento generale dei tempi di ETL.
	Quindi in caso di aumento notevole delle dimensioni del database potrebbe
	risultare necessario, per soddisfare le deadline dell’ETL, utilizzare uno schema
	senza i dati precalcolati, o almeno mantenere come precalcolati solo quelli più
	utilizzati.
	\item Quando una gerarchia è basata su unità di misura fisiche (ad esempio distanza o
	tempo), può convenire effettuare operazioni di raggruppamento su espressioni
	calcolabili sui fatti, invece che utilizzare l’implementazione tradizionale che
	prevede una tabella separata per ogni gerarchia e utilizzo di join, che possono
	appesantire notevolmente il calcolo.
	\item Dal punto di vista dello spazio fisico occupato è stato osservato che, con circa due
	milioni di registrazioni nella tabella dei fatti, memorizzare i dati precalcolati porta
	ad un aumento di circa il 17.5\% (da 639 a 744 MB). In un contesto di Data
	Warehouse un incremento del genere non può essere comunque considerato
	critico, soprattutto rispetto al guadagno in termini di tempo.
	\item L’utilizzo della vista materializzata ha mostrato, come previsto, un vantaggio
	notevole, con un rallentamento accettabile della fase di ETL di aggiornamento
	delle viste.
	\item Il partizionamento non ha mostrato evidenti miglioramenti che ne giustifichino
	l’utilizzo. Infatti, si è registrato un peggioramento netto della fase di ETL (in cui la
	durata dei passi 3 e 4 triplica) e delle query che prevedono la ricostruzione parziale
	o totale del fatto, avvantaggiando solamente (e naturalmente) le query circoscritte
	ad una singola partizione. Tuttavia, prima di screditare completamente questa
	tecnica, bisognerebbe prendere in considerazione che questa andrebbe valutata
	in contesti in cui il numero e l’importanza di query circoscritte a singole partizioni
	potrebbe effettivamente portare a dei benefici.
\end{itemize}