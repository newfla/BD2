\chapter{Conclusioni}
I risultati ottenuti sono chiaramente legati all'architettura hardware e software utilizzata per l'esecuzione delle procedure. Durante l'intero progetto è stata utilizzata una piattaforma con le seguenti specifiche:
\begin{itemize}
	\item Xiaomi Notebook Air 13
	\item Intel Core i7
	\item SSD da 256GB
	\item RAM da 8GB
	\item Windows 10
\end{itemize}
I tempi sono necessariamente soggetti a limiti fisici e rumori dovuti a:
\begin{enumerate}
	\item Capacità hardware del calcolatore
	\item Scheduling del sistema operativo.
	\item Parametri di configurazione del DBMS.
\end{enumerate}
Per ammortizzare la perturbazione, le fasi di cronometraggio prevedono l'acquisizione di più campioni dello stesso tipo, dai quali si è poi ricavata la media. I valori ottenuti vanno inoltre scalati rispetto alla dimensione del dataset. A tal proposito si è simulata una cardinalità massiccia di acquisizioni (circa 1.5 milioni).\\
A valle dell’analisi condotta si è arrivati alle seguenti riflessioni:
%- Aumentare la quantità di memoria principale a disposizione per le operazioni di raggruppamento, ordinamento e join può sensibilmente migliorare l’esecuzione di alcune query, permettendo l’utilizzo di operazioni completamente in-memory.

\begin{itemize}
	\item Quando una gerarchia coincide con scalare un'unità di misura ci si può imbattere in allocazioni massicce di informazioni ridondanti, senza un gadagno netto in termini temporali. In questo scenario si può quindi optare per operazioni di raggruppamento su espressioni	calcolabili sui fatti, invece che utilizzare l’implementazione tradizionale che prevede una tabella separata per ogni gerarchia e utilizzo di join, che possono appesantire notevolmente il calcolo.
	\item L’utilizzo della vista materializzata ha mostrato, come previsto, un vantaggio notevole, con un rallentamento accettabile della fase di ETL di aggiornamento delle viste.
	\item Il partizionamento non ha mostrato evidenti miglioramenti che ne giustifichino l’utilizzo. Infatti, si è registrato un peggioramento netto della fase di ETL e delle query che prevedono la ricostruzione del fatto, avvantaggiando solamente (e naturalmente) le query circoscritte ad una singola partizione. Tuttavia, prima di screditare completamente questa tecnica, bisognerebbe prendere in considerazione che questa andrebbe valutata
	in contesti in cui il numero e l’importanza di query circoscritte a singole partizioni potrebbe effettivamente portare a dei benefici.
\end{itemize}